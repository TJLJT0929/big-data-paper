% !TEX program = xelatex
\documentclass[hyperref,a4paper,UTF8]{ctexart}

\usepackage[left=2.50cm, right=2.50cm, top=2.50cm, bottom=2.50cm]{geometry}

\usepackage{setspace}
\onehalfspacing

\usepackage[unicode=true,colorlinks,urlcolor=blue,linkcolor=blue,bookmarksnumbered=true]{hyperref}
\usepackage{latexsym,amssymb,amsmath,amsbsy,amsopn,amstext,amsthm,amsxtra,color,bm,calc,ifpdf}
\usepackage{graphicx}
\usepackage{subcaption}
\usepackage{enumerate}
\usepackage{fancyhdr}
\usepackage{algorithm}
\usepackage{algpseudocode}
\usepackage{makecell}
\usepackage{float}
\usepackage{listings}
\usepackage{multirow}
\usepackage[numbers,sort&compress]{natbib}
\usepackage{makeidx}
\usepackage{xcolor}
\usepackage{booktabs} 
\usepackage{fontspec}
\usepackage{hyperref}
\usepackage{pythonhighlight}
\graphicspath{{figures/}}


\newcommand{\upcite}[1]{\textsuperscript{\cite{#1}}}
\newtheorem{theorem}{Theorem}
\newtheorem{assumption}{Assumption}
\newtheorem{lemma}{Lemma}

\providecommand{\tightlist}{%
	\setlength{\itemsep}{0pt}\setlength{\parskip}{0pt}}

\pagestyle{fancy}
\fancyhead[L]{}
\fancyhead[C]{\fangsong 基于LendingClub平台数据的贷款违约预测研究}
\fancyhead[R]{}

\renewcommand{\abstractname}{\textbf{\large {摘\quad 要}}} 


\title{\textbf{{基于LendingClub平台数据的贷款违约预测研究}}\\
}
\author{
	\kaishu\normalsize
	姓名\ \underline{  刘建廷  } \qquad
	学号\ \underline{  251601032  } \qquad
	院系\ \underline{  中南大学商学院  }
}



\begin{document}
	
	\maketitle
	
	\vspace{0.5cm}
	
	\begin{abstract}
		近年来P2P网络借贷业务规模持续扩大,随之而来的信用风险管理难题亦愈发凸显。本文以美国LendingClub平台公开披露的贷款数据(约39.6万条记录、28个特征变量)为研究对象,结合探索性数据分析、特征工程以及多种机器学习算法,对借款人的违约行为展开建模与预测研究。探索性分析环节借助描述性统计和Pearson相关性检验,筛选出贷款利率、信用等级、债务收入比、贷款期限等若干核心风险因子。建模环节分别搭建了人工神经网络(ANN)、极端梯度提升(XGBoost)及随机森林三类分类器,采用准确率、精确率、召回率、F1值和ROC曲线下面积(AUC)等指标对各模型的预测效果加以衡量。实验结果显示,ANN的测试集AUC达0.905,较XGBoost的0.734和随机森林的0.725有明显提升,在违约识别的综合判别力方面表现最优。上述发现可为P2P平台优化风控策略和改进贷款审批流程提供定量层面的决策参考。
	\end{abstract}
	
	\textbf{关键词:}贷款违约预测;LendingClub;人工神经网络;XGBoost;随机森林
	
	\section{引言}
	
	\subsection{研究背景}
	
	金融科技(FinTech)的快速发展正在重塑传统金融服务格局。P2P(Peer-to-Peer)网络借贷是这一变革中颇具代表性的业务模式——它依托互联网平台将资金供给方与需求方直接对接,省去了传统银行体系中的多层中介环节,从而降低了融资成本并提高了资金周转速度。LendingClub(借贷俱乐部)2006年创立于美国旧金山,是全球最早且规模较大的P2P借贷平台之一。该平台率先在美国证券交易委员会(SEC)完成证券登记,其业务覆盖个人消费贷款、小微企业贷款以及医疗融资等类别。
	
	不过,P2P模式在拓宽融资渠道的同时也引入了显著的信用风险敞口。对平台而言,这一风险具有双向性:一方面,若将具备还款能力的优质借款人误判为高风险而予以拒绝,则平台将面临客户流失和潜在收益损失;另一方面,若向偿付能力不足的申请人放款,所产生的坏账将直接侵蚀平台资本与投资者收益。正因如此,准确识别潜在违约人群、建立可靠的信用风险评估模型,一直是P2P借贷研究与实务中的核心问题。
	
	\subsection{研究现状}
	
	贷款违约预测是信用风险管理领域的经典命题,相关文献积累较为丰厚。在传统统计方法方面,Logistic回归因参数估计高效、模型结构透明等优点,长期在信用评分实务中占据主流地位\cite{thomas2002credit}。但在面对高维特征空间与变量间错综复杂的非线性映射时,线性模型的拟合能力显得力不从心。
	
	随着机器学习算法体系日趋完善,不少研究者尝试将其引入信用风险评估领域。 Lessmann等\cite{lessmann2015benchmarking}曾对多类分类算法在信用评分场景中的表现做过系统性比较,结果显示以随机森林和梯度提升树为代表的集成方法在预测精度上普遍优于单一分类器。Xia等\cite{xia2017boosted}将贝叶斯超参数优化策略引入XGBoost信用评分框架,在LendingClub数据上取得了较好的判别效果。深度学习方向上,Kvamme等\cite{kvamme2018predicting}的工作表明深层神经网络在捕获复杂非线性特征交互方面展现出独特的优势。
	
	在P2P借贷实证文献中,LendingClub公开数据集因样本量大、特征覆盖面广、数据质量相对可控而被学界频繁使用。Serrano-Cinca和Gutiérrez-Nieto\cite{serrano2016use}利用该数据集考察了违约行为的影响因素,研究指出贷款等级、利率以及借款人收入水平是预测违约的关键变量。Malekipirbazari与Aksakalli\cite{malekipirbazari2015risk}在同一平台数据上比较了多种机器学习分类器,其中随机森林在多个评估指标上胜出。
	
	纵观已有文献,以下几个方面仍有进一步探索的余地:其一,大多数工作仅关注单一或少量模型,缺少传统树模型与深度神经网络之间的系统性横向比较;其二,各研究在数据清洗策略和特征构造方案上差异较大,对异常值的检测与处理缺乏统一规范;其三,围绕样本类别不均衡问题的讨论仍不够深入。
	
	\subsection{研究目的与意义}
	
	鉴于上述不足,本文选取LendingClub平台贷款数据作为实证样本,研究工作围绕四个方面展开:(1)通过探索性数据分析,厘清贷款违约行为的主要影响因素及其统计分布特征;(2)设计规范化的数据预处理和特征工程流程,形成适宜建模的高质量样本集;(3)分别搭建人工神经网络、XGBoost以及随机森林三类预测模型,并就其分类性能开展多维指标比较;(4)为P2P借贷平台改进贷款审批决策与风险防控机制提供数据层面的参考。
	
	就理论贡献而言,本文将深度学习模型引入P2P信用评估的比较分析中,拓展了该领域的方法论视野,也为后续研究的模型选型提供了实证依据。就实践价值而言,研究结论可为LendingClub等平台的风控策略制定和审批效率提升提供技术参考。
	
	
	\section{数据来源与分析方法}
	
	\subsection{数据来源}
	
	本文所用数据集名为``All Lending Club loan data'',来源于Kaggle开放数据平台\cite{kaggle_lendingclub},涵盖了LendingClub自2007年开展业务以来所有获批贷款和被拒贷款的完整记录。
	
	该数据集共计约396{,}030条贷款记录,包含28个特征变量,基本覆盖了该平台信贷业务的主要维度。每一条记录对应一笔贷款申请,因变量为贷款状态(\texttt{loan\_status}),分为``Fully Paid''(全额偿还)与``Charged Off''(违约核销)两类。从原始样本的分布来看,全额偿还类约占80.38\%,违约核销类约占19.62\%,二者比例约为4:1,体现出信用风险建模中典型的类别失衡特征。
	
	\subsection{数据特征描述}
	
	数据集中的28个自变量按照业务含义大致划分为三类。
	
	\subsubsection{借款人基本信息特征}
	
	该类特征(见表~\ref{tab:borrower_info})主要描述借款人的就业状况、住房条件、收入水平及信用历史开立时间等基本信息。
	
	\begin{table}[htbp]
		\centering
		\caption{借款人基本信息特征}\label{tab:borrower_info}
		\begin{tabular}{ll}
			\toprule
			变量名 & 描述 \\
			\midrule
			\texttt{emp\_title}         & 借款人提供的职位名称 \\
			\texttt{emp\_length}        & 就业年限(0--10年,10表示十年及以上) \\
			\texttt{home\_ownership}    & 住房所有权状况(RENT/OWN/MORTGAGE/OTHER) \\
			\texttt{annual\_inc}        & 借款人自报年收入 \\
			\texttt{verification\_status} & 收入验证状态 \\
			\texttt{address}            & 借款人地址 \\
			\texttt{earliest\_cr\_line} & 最早信用额度开立月份 \\
			\bottomrule
		\end{tabular}
	\end{table}
	
	\subsubsection{贷款属性特征}
	
	该类特征(见表~\ref{tab:loan_attr})包括贷款金额、期限、利率和信用等级等与贷款合约直接相关的字段。
	
	\begin{table}[htbp]
		\centering
		\caption{贷款属性特征}\label{tab:loan_attr}
		\begin{tabular}{ll}
			\toprule
			变量名 & 描述 \\
			\midrule
			\texttt{loan\_amnt}           & 贷款申请金额 \\
			\texttt{term}                 & 贷款期限(36个月或60个月) \\
			\texttt{int\_rate}            & 贷款利率 \\
			\texttt{installment}          & 月还款金额 \\
			\texttt{grade}                & 平台分配的贷款等级(A--G) \\
			\texttt{sub\_grade}           & 平台分配的贷款子等级 \\
			\texttt{issue\_d}             & 贷款发放月份 \\
			\texttt{purpose}              & 贷款用途类别 \\
			\texttt{title}                & 借款人提供的贷款标题 \\
			\texttt{initial\_list\_status} & 贷款初始上市状态(W/F) \\
			\texttt{application\_type}    & 申请类型(个人/联合申请) \\
			\bottomrule
		\end{tabular}
	\end{table}
	
	\subsubsection{借款人信用状况特征}
	
	该类特征(见表~\ref{tab:credit_info})刻画了借款人在信用体系中的负债状况、账户结构及不良信用记录等信息。
	
	\begin{table}[htbp]
		\centering
		\caption{借款人信用状况特征}\label{tab:credit_info}
		\begin{tabular}{ll}
			\toprule
			变量名 & 描述 \\
			\midrule
			\texttt{dti}                    & 债务收入比(月债务支出/月收入) \\
			\texttt{open\_acc}              & 开放信用额度数 \\
			\texttt{pub\_rec}               & 不良公共记录数 \\
			\texttt{revol\_bal}             & 循环信用总余额 \\
			\texttt{revol\_util}            & 循环信用利用率 \\
			\texttt{total\_acc}             & 总信用额度数 \\
			\texttt{mort\_acc}              & 抵押贷款账户数 \\
			\texttt{pub\_rec\_bankruptcies} & 公共记录中的破产次数 \\
			\bottomrule
		\end{tabular}
	\end{table}
	
	\subsection{研究方法与技术路线}
	
	整个研究按先后顺序分为以下四个阶段。
	
	\textbf{第一阶段:探索性数据分析。}
	利用描述性统计、可视化图表(直方图、箱线图、热力图、计数图等)和Pearson相关系数,对各特征的分布形态、变量之间的关联结构以及各特征与因变量的关系强度进行逐一考察。具体分析内容包括因变量的类别分布、各自变量的单变量与双变量统计、特征间相关性热力图以及数值型特征与因变量的相关性排序。
	
	\textbf{第二阶段:数据预处理与特征工程。}
	原始数据存在若干质量问题,需要经过一系列处理方可用于建模。缺失值方面,对于缺失比例较低的字段(如\texttt{revol\_util}和\texttt{pub\_rec\_bankruptcies},缺失率均不到0.5\%),直接剔除相应样本即可;缺失比例较高的\texttt{mort\_acc}字段,则根据相关性分析结果,以与之关联最强的\texttt{total\_acc}为分组依据,按组均值补齐。特征筛选方面,剔除了唯一值过多、建模价值有限的\texttt{emp\_title},以及可能导致数据泄露的\texttt{issue\_d}等时间变量;由于各就业年限组的违约率差异极小,\texttt{emp\_length}也予以删除;\texttt{title}与\texttt{purpose}信息高度重合,仅保留后者。异常值方面,依据探索性分析的统计发现,在训练集中剔除年收入超过250{,}000美元、DTI超过50等极端记录。编码方面,对\texttt{sub\_grade}、\texttt{verification\_status}、\texttt{purpose}等类别变量施加独热编码。此外还涉及若干转换操作:\texttt{term}由字符串映射为数值、\texttt{address}中提取邮政编码、\texttt{earliest\_cr\_line}中提取年份。所有特征最终经MinMaxScaler归一化至$[0, 1]$区间,以消除量纲差异对训练过程的干扰。
	
	\textbf{第三阶段:模型构建与训练。}
	数据集按67:33的比例拆分为训练集(264{,}796条)和测试集(130{,}423条),训练集经异常值剔除后剩余262{,}143条,最终特征维度为79。本文构建了以下三种分类模型。
	
	(1)人工神经网络(ANN)。基于Keras框架搭建前馈网络,设置3个隐藏层(每层150个神经元),隐藏层激活函数为ReLU,输出层使用Sigmoid以输出违约概率。训练时加入批标准化(Batch Normalization)与Dropout(丢弃率0.1)来抑制过拟合。优化器为Adam(学习率0.001),损失函数选用二元交叉熵,共迭代25个epoch,batch size设为256。(2)XGBoost分类器。调用XGBoost库的\texttt{XGBClassifier}接口,沿用默认超参数。XGBoost属于梯度提升框架下的集成方法,核心思路是逐轮构建决策树来拟合前一轮预测的残差,再将各弱学习器的输出加权集成\cite{chen2016xgboost}。(3)随机森林分类器。使用Scikit-learn的\texttt{RandomForestClassifier},基学习器数目设为100。随机森林借助自助采样(Bootstrap Sampling)和随机特征子集抽取构建决策树集合,最终通过多数投票决定预测类别\cite{breiman2001random}。
	
	\textbf{第四阶段:模型评估与比较。}
	为了较为全面地评价三种模型的分类效果,本文选用了多个维度的评价指标。准确率(Accuracy)衡量被正确分类样本占总体的比例;精确率(Precision)关注的是模型判为正类的样本中实际为正类的占比;召回率(Recall)则反映实际正类样本中有多少被模型成功捕获;F1值是精确率与召回率的调和均值,对二者做了综合平衡;ROC-AUC(受试者工作特征曲线下面积)从整体上度量模型区分正负样本的能力,在类别不均衡条件下尤其具有参考意义。此外还引入混淆矩阵,将各类别的预测情况以矩阵形式呈现,方便对模型在不同类别上的表现做细粒度审视。
	
	\subsection{实验环境}
	
	实验所用编程语言为Python 3,数据读取和处理主要依赖Pandas和NumPy库,可视化部分用到了Matplotlib、Seaborn以及hvPlot。机器学习算法的实现基于Scikit-learn\cite{pedregosa2011scikit}和XGBoost库,深度学习部分采用TensorFlow/Keras框架,全部实验在Kaggle Notebook环境中完成。
	
	
	\section{研究结果与分析}
	
	\subsection{探索性数据分析结果}
	
	\subsubsection{因变量分布}
	
	对因变量\texttt{loan\_status}做频率统计可知,全额偿还(Fully Paid)类占80.38\%,违约核销(Charged Off)类占19.62\%,二者之比接近4:1。这是一个在信用风险建模中相当常见的类别失衡格局。值得注意的是,一个将全部样本判为多数类的"朴素分类器"仅凭这种分布就能获得近80\%的准确率,因此后续建模时有必要采用ROC-AUC等更具鉴别力的指标,同时关注类别不均衡对训练过程的潜在影响。

	
	\subsubsection{特征间相关性分析}
	
	图~\ref{fig:corr_heatmap} 为数值型特征之间的Pearson相关系数热力图。从中可以归纳出若干值得关注的发现。
	
	贷款金额(\texttt{loan\_amnt})与月还款额(\texttt{installment})几乎完全正相关(相关系数接近1.0),这在业务逻辑上很好理解——月还款额本身就是由贷款金额、利率和期限共同决定的,二者之间的信息冗余度很高。贷款金额与利率之间也呈正相关,大额贷款往往对应更高的利率定价。开放信用额度数(\texttt{open\_acc})和总信用额度数(\texttt{total\_acc})的正相关关系也比较明显。另外,抵押贷款账户数(\texttt{mort\_acc})与总信用额度数之间的相关性在所有变量对中最为突出,这一发现为后面设计缺失值填补方案提供了依据。
	
	\begin{figure}[H]
		\centering
		\includegraphics[width=0.9\textwidth]{Figure/2}
		\caption{数值型特征间 Pearson 相关系数热力图}
		\label{fig:corr_heatmap}
	\end{figure}
	
	\subsubsection{贷款金额与月还款额分析}
	
	将贷款金额和月还款额按贷款状态做分层比较,结果见图~\ref{fig:loan_amnt_installment}。无论从直方图还是箱线图来看,违约组的贷款金额中位数和均值都略高于全额偿还组,暗示贷款金额偏高可能与违约风险有一定关联。月还款额的分布走势与贷款金额大体一致,这再次印证了二者的高度相关性。从箱线图更具体地看,违约组在四分位间距和极值范围上均高于偿还组,也就是说违约借款人往往倾向于申请金额较大的贷款。
	


\begin{figure}[H]
	\centering
	
	\begin{subfigure}[b]{0.48\textwidth}
		\centering
		\includegraphics[width=\linewidth]{Figure/3a}
		\caption{贷款金额直方图}
		\label{fig:loan_amnt_hist}
	\end{subfigure}
	\hfill
	\begin{subfigure}[b]{0.48\textwidth}
		\centering
		\includegraphics[width=\linewidth]{Figure/3b}
		\caption{月还款额直方图}
		\label{fig:installment_hist}
	\end{subfigure}
	
	\vspace{0.5cm}
	
	\begin{subfigure}[b]{0.42\textwidth}
		\centering
		\includegraphics[width=\linewidth]{Figure/3c}
		\caption{贷款金额箱线图}
		\label{fig:loan_amnt_box}
	\end{subfigure}
	\hfill
	\begin{subfigure}[b]{0.42\textwidth}
		\centering
		\includegraphics[width=\linewidth]{Figure/3d}
		\caption{月还款额箱线图}
		\label{fig:installment_box}
	\end{subfigure}
	
	\caption{贷款金额与月还款额按贷款状态的分布}
	\label{fig:loan_amnt_installment}
\end{figure}
	
	\subsubsection{贷款等级与子等级分析}
	
	LendingClub按照借款人信用状况,将贷款分为A(最优)到G(最差)七档等级。图~\ref{fig:grade_subgrade_a} 给出了各等级和子等级的贷款状态分布情况。数据呈现出一条清晰的违约梯度:A等级的违约占比最低,随着等级下降到F和G,违约占比逐步攀升。图~\ref{fig:grade_subgrade_b} 更细致地展示了F、G两档子等级内部的分布,可以发现违约样本与偿还样本的数量差距大幅收窄,个别子等级的违约率甚至逼近50\%。这说明LendingClub的内部评级体系对借款人的违约风险具有较强的区分能力,评级信息在信用风险建模中具有不可忽视的参考价值。
	
\begin{figure}[H]
	\centering
	
	\begin{subfigure}[b]{0.4\textwidth}
		\centering
		\includegraphics[width=\linewidth]{Figure/4a}
		\caption{按贷款等级与子等级划分的贷款状态分布}
		\label{fig:grade_subgrade_a}
	\end{subfigure}
	\hfill
	\begin{subfigure}[b]{0.45\textwidth}
		\centering
		\includegraphics[width=\linewidth]{Figure/4b}
		\caption{F和G等级贷款的违约分布}
		\label{fig:grade_subgrade_b}
	\end{subfigure}
	
	\caption{贷款等级与违约状态分析}
	\label{fig:grade_subgrade}
\end{figure}
	
	\subsubsection{贷款期限、住房状况、收入验证与贷款用途分析}
	
	图~\ref{fig:term_home_verify_purpose} 展示了贷款期限、住房所有权、收入验证状态和贷款用途四个类别特征的分层分布。贷款期限方面,60个月期限的违约率明显高于36个月,这表明还款周期拉长带来的不确定性的确会推高违约概率。住房所有权方面,租房(RENT)和按揭(MORTGAGE)群体占据了样本的绝大部分,各类别之间的违约率差异不太突出;``ANY''和``NONE''两类样本量极少、统计稳定性差,本文将其合并到``OTHER''中。收入验证状态方面,Verified、Source Verified和Not Verified三组借款人的违约比例相差无几,说明这一变量对违约预测的增量贡献十分有限。贷款用途方面,债务整合(debt\_consolidation)是最常见的借款原因,信用卡偿还(credit\_card)排在其次;不同用途间虽存在一定的违约率差异,但分化幅度远不如利率和信用等级那样显著。
	
	\begin{figure}[H]
		\centering
		\includegraphics[width=0.85\textwidth]{Figure/5}
		\caption{贷款期限、住房状况、收入验证状态与贷款用途按贷款状态的分布}
		\label{fig:term_home_verify_purpose}
	\end{figure}
	
	\subsubsection{利率与年收入分析}
	
	贷款利率与年收入是衡量借款人信用风险水平的两个关键变量。从图~\ref{fig:int_rate} 可以看到,违约样本的利率分布明显右偏、集中在高利率区间,偿还样本则偏向低利率区间,两组的分布差异相当显著。这背后的经济学逻辑并不复杂:高利率往往意味着平台对该借款人的信用风险评级较高,与此同时较高的利率也加大了每月的偿还压力,两方面因素叠加使得违约概率上升。
	
	年收入的分布(图~\ref{fig:annual_inc})显示,绝大多数借款人的年收入在250{,}000美元以下。进一步统计发现,年收入超过1{,}000{,}000美元的仅有75人,不到总样本的0.02\%,超过250{,}000美元的也仅4{,}077人。整体上看,高收入人群的违约率偏低,但该群体样本量太小,据此得出的统计推断需要谨慎对待。
	
\begin{figure}[htbp]
	\centering
	
	\begin{subfigure}[b]{0.4\textwidth}
		\centering
		\includegraphics[width=\linewidth]{Figure/6a}
		\caption{贷款利率按贷款状态的分布直方图}
		\label{fig:int_rate}
	\end{subfigure}
	\hfill
	\begin{subfigure}[b]{0.55\textwidth}
		\centering
		\includegraphics[width=\linewidth]{Figure/6b}
		\caption{年收入($\leq$250{,}000美元)按贷款状态的分布直方图}
		\label{fig:annual_inc}
	\end{subfigure}
	
	\caption{贷款利率与年收入按贷款状态的分布}
	\label{fig:int_rate_annual_inc}
\end{figure}
	
	\subsubsection{债务收入比与循环信用分析}
	
	图~\ref{fig:dti_revol} 分别给出了债务收入比、循环信用利用率和循环信用余额三个指标按贷款状态的分布图。债务收入比方面,DTI越高的借款人违约倾向越明显,不过DTI超过50的极端观测极少,后续建模时作为异常值剔除。循环信用利用率方面,利用率偏高的借款人违约风险也相应上升,超过120\%的记录极为罕见,同样视为异常值。循环信用余额方面,超过250{,}000美元的样本寥寥无几,其中也夹杂着违约案例。综合来看,借款人的负债水平与信用使用状况对其违约行为有着较强的解释力,是风险评估中不可忽视的参考维度。
	
\begin{figure}[H]
	\centering
	\begin{subfigure}[b]{0.32\textwidth}
		\centering
		\includegraphics[width=\linewidth]{Figure/7a}
	\end{subfigure}
	\hfill
	\begin{subfigure}[b]{0.32\textwidth}
		\centering
		\includegraphics[width=\linewidth]{Figure/7b}
	\end{subfigure}
	\hfill
	\begin{subfigure}[b]{0.32\textwidth}
		\centering
		\includegraphics[width=\linewidth]{Figure/7c}
	\end{subfigure}
	
	\caption{债务收入比、循环信用利用率及循环信用余额按贷款状态的分布}
	\label{fig:dti_revol}
\end{figure}

	
	\subsubsection{公共记录与破产记录分析}
	
	图~\ref{fig:pub_rec_bankruptcy} 呈现了不良公共记录、初始上市状态、申请类型、破产记录及抵押贷款账户数等变量的分层分布。不良公共记录方面,绝大多数借款人的\texttt{pub\_rec}为0,而有不良记录的借款人对应的违约率确实偏高。破产记录呈现类似的规律——无破产历史的借款人占大多数,有过破产经历者违约风险有所抬升。据此,在特征工程环节将不良记录与破产记录各自转化为0/1二值变量,以简化表示并突出其"有无"对违约概率的区分作用。初始上市状态和申请类型方面,各类别间的违约率差异相对不大。
	
	\begin{figure}[H]
		\centering
		\includegraphics[width=0.9\textwidth]{Figure/8}
		\caption{公共记录、初始上市状态、申请类型、破产记录与抵押贷款账户按贷款状态的分布}
		\label{fig:pub_rec_bankruptcy}
	\end{figure}
	
	\subsubsection{数值特征与因变量的相关性排序}
	
	将所有数值型特征与因变量的Pearson相关系数按绝对值降序排列,结果见图~\ref{fig:feature_corr_target}。与偿还行为正相关的特征中,抵押贷款账户数和年收入的正向作用幅度较小,说明单靠收入或资产指标还不足以有效分开违约和非违约群体。另一方面,与违约行为正相关(也就是与偿还行为负相关)的特征中,利率(\texttt{int\_rate})的关联程度最为突出,贷款期限(\texttt{term})和债务收入比(\texttt{dti})紧随其后。这与前面探索性分析中得到的定性结论相互印证,同时也为后续建模中的特征筛选和结果解读提供了定量参考。
	
	\begin{figure}[H]
		\centering
		\includegraphics[width=0.9\textwidth]{Figure/9}
		\caption{数值型特征与贷款状态(因变量)的Pearson相关系数排序}
		\label{fig:feature_corr_target}
	\end{figure}
	
	\subsection{数据预处理结果}
	
	\subsubsection{缺失值处理}
	
	原始数据中存在缺失值的特征及对应处理方式汇总于表~\ref{tab:missing}。
	
	\begin{table}[H]
		\centering
		\caption{缺失值处理方案}\label{tab:missing}
		\begin{tabular}{llll}
			\toprule
			特征 & 缺失数量 & 缺失比例 & 处理方式 \\
			\midrule
			\texttt{emp\_title}            & 较多 & --- & 删除该特征(唯一值过多,不适合编码) \\
			\texttt{emp\_length}           & 较多 & --- & 删除该特征(各就业年限违约率几乎一致) \\
			\texttt{title}                 & 少量 & --- & 删除该特征(与\texttt{purpose}信息重叠) \\
			\texttt{mort\_acc}             & 较多 & --- & 按\texttt{total\_acc}分组均值填补 \\
			\texttt{revol\_util}           & 少量 & $<$0.5\% & 删除含缺失值的行 \\
			\texttt{pub\_rec\_bankruptcies} & 少量 & $<$0.5\% & 删除含缺失值的行 \\
			\bottomrule
		\end{tabular}
	\end{table}
	
	经以上处理后数据集中绝大部分样本得以保留,因缺失值清理而损失的信息量可以忽略。
	
	\subsubsection{特征工程}
	
	特征工程完成后,数据集的特征维度达到79(含独热编码生成的虚拟变量)。处理步骤如下:先将\texttt{term}从字符型(`` 36 months''/`` 60 months'')转为对应数值(36/60),方便模型直接利用其数值信息。因为\texttt{sub\_grade}已经包含了\texttt{grade}的全部信息(子等级即等级的细分),故删去\texttt{grade}以避免冗余。对\texttt{sub\_grade}、\texttt{verification\_status}、\texttt{purpose}、\texttt{initial\_list\_status}、\texttt{application\_type}和\texttt{home\_ownership}等类别型变量做独热编码,设置drop\_first=True以规避多重共线性。\texttt{address}中提取邮政编码后做独热编码,\texttt{earliest\_cr\_line}中提取年份作为数值特征。\texttt{issue\_d}予以删除,避免使用贷款发放日期这一未来信息造成数据泄露。另外,\texttt{pub\_rec}、\texttt{mort\_acc}和\texttt{pub\_rec\_bankruptcies}被转化为0/1二值变量(0表示无记录、1表示有记录),以突出"有无"这一属性对违约风险的区分意义。
	
	\subsubsection{异常值处理}
	
	基于探索性分析阶段的统计发现,在训练集中按表~\ref{tab:outlier} 所列规则剔除异常样本。
	
	\begin{table}[htbp]
		\centering
		\caption{异常值剔除规则}\label{tab:outlier}
		\begin{tabular}{ll}
			\toprule
			过滤条件 & 说明 \\
			\midrule
			\texttt{annual\_inc} $\leq$ 250{,}000    & 移除极端高收入样本 \\
			\texttt{dti} $\leq$ 50                    & 移除极端高债务收入比样本 \\
			\texttt{open\_acc} $\leq$ 40              & 移除开放信用账户数过多的样本 \\
			\texttt{total\_acc} $\leq$ 80             & 移除总信用账户数过多的样本 \\
			\texttt{revol\_util} $\leq$ 120           & 移除循环信用利用率过高的样本 \\
			\texttt{revol\_bal} $\leq$ 250{,}000      & 移除循环信用余额过高的样本 \\
			\bottomrule
		\end{tabular}
	\end{table}
	
	这里需要特别说明的是,异常值剔除只针对训练集执行,测试集保持原样不动。这样做是为了让模型在贴近真实数据分布的条件下接受检验,从而更客观地反映泛化性能。剔除后训练集从264{,}796条减少到262{,}143条,比例约1.0\%。
	
	\subsection{模型预测结果}
	
\subsubsection{人工神经网络模型结果}

ANN在训练集和测试集上均表现稳定:训练集准确率88.84\%,测试集88.87\%,两者几乎没有落差,未出现明显过拟合。该模型对全额偿还类(类别1)的识别效果突出,召回率高达0.99;但对违约类(类别0)的召回率仅0.46,类别不均衡的影响在此暴露得比较充分。两类的F1值分别为0.62和0.93。混淆矩阵的细节也印证了这一点——训练集中51,665个实际违约样本里有27,985个被误判为偿还,测试集中25,480个违约样本里误判数达13,798个。

图~\ref{fig:ann_training}给出了训练过程中损失函数和AUC随epoch变化的曲线。损失值在最初几个轮次内快速下降后趋于平稳,AUC则同步上升并逐渐收敛,整个训练过程比较平稳。最终,ANN的ROC-AUC得分为0.905,在三种模型中排在第一位,综合判别正负样本的能力最强。

\begin{figure}[H]
	\centering
	\includegraphics[width=0.9\textwidth]{Figure/10}
	\caption{ANN模型训练过程中的损失函数值与AUC指标演化曲线}
	\label{fig:ann_training}
\end{figure}
	
\subsubsection{XGBoost分类器结果}

XGBoost的整体表现比较稳健——训练集准确率89.60\%,测试集88.94\%,差距甚微,泛化能力令人满意。违约类的识别方面,训练集和测试集召回率分别为0.50和0.48,均略优于ANN,对少数类的捕获能力稍好一些。全额偿还类的召回率依然维持在0.99的高水平。不过,该模型的ROC-AUC仅为0.734,与ANN的0.905差距较大,在综合区分正负样本方面显得不足。

从训练集混淆矩阵看,51,665个实际违约样本中模型正确识别了25,828个,另有25,837个被误判为偿还类。测试集的具体分类情况见图~\ref{fig:xgb_cm}中的混淆矩阵。
	
\subsubsection{随机森林分类器结果}

随机森林出现了较为典型的过拟合情况。训练集上,该模型达到了"完美"的分类效果——准确率100\%,各类精确率、召回率及F1值均为1.00,混淆矩阵中无一例误判。然而到了测试集,性能明显下滑,准确率降至88.94\%。违约类召回率为0.46,和ANN相当,但精确率高达0.96;全额偿还类召回率同样保持在0.99。ROC-AUC得分0.725,在三种模型中垫底。训练与测试之间如此悬殊的性能落差表明,随机森林在训练阶段很可能因为树结构过深而"记住"了训练样本的细节,导致泛化到新数据时表现打折。测试集的误判细节可参见图~\ref{fig:rf_cm}的混淆矩阵。

\begin{figure}[H]
	\centering
	\begin{minipage}{0.48\textwidth}
		\centering
		\includegraphics[width=\linewidth]{Figure/11}
		\caption{XGBoost模型测试集混淆矩阵}
		\label{fig:xgb_cm}
	\end{minipage}
	\hfill
	\begin{minipage}{0.48\textwidth}
		\centering
		\includegraphics[width=\linewidth]{Figure/12}
		\caption{随机森林模型测试集混淆矩阵}
		\label{fig:rf_cm}
	\end{minipage}
\end{figure}

	
	\subsubsection{模型综合比较}
	
	表~\ref{tab:model_compare} 汇总了三种模型在测试集上的核心性能指标,图~\ref{fig:model_compare}则以柱状图形式直观呈现了各模型在训练集和测试集上的ROC-AUC得分差异。
	
	\begin{table}[htbp]
		\centering
		\caption{三种模型性能综合对比}\label{tab:model_compare}
		\begin{tabular}{lccc}
			\toprule
			模型 & 训练集准确率 & 测试集准确率 & 测试集ROC-AUC \\
			\midrule
			人工神经网络(ANN) & 88.84\% & 88.87\% & \textbf{0.905} \\
			XGBoost             & 89.60\% & 88.94\% & 0.734 \\
			随机森林            & 100.00\% & 88.94\% & 0.725 \\
			\bottomrule
		\end{tabular}
	\end{table}
	
	\begin{figure}[H]
		\centering
		\includegraphics[width=0.75\textwidth]{Figure/13}
		\caption{三种模型在训练集与测试集上的ROC-AUC得分对比}
		\label{fig:model_compare}
	\end{figure}
	
	从ROC-AUC来看,ANN以0.905的得分大幅领先XGBoost的0.734与随机森林的0.725,领先幅度分别约17和18个百分点,。这一差距说明神经网络在综合区分正负样本方面占据明显优势,对贷款数据中复杂的非线性特征交互关系的捕捉更为充分。
	
	准确率方面,三种模型测试集成绩均在89\%上下浮动,看上去差别不大。但考虑到非违约样本本身就占了约80\%,准确率的参考价值打了折扣——一个“全部预测为偿还”的极端策略就能拿到约80\%的准确率。由此可见,ROC-AUC才是评判模型优劣时更值得信赖的综合指标。
	
	过拟合程度方面,ANN在训练集与测试集上的表现几乎无差异,泛化最为稳健,这主要得益于批标准化和Dropout这两项正则化手段。XGBoost存在轻微过拟合但在可接受范围内,梯度提升框架自带的正则化机制(树深限制、学习率衰减等)起了一定的约束作用。随机森林的过拟合最为严重——训练集100\%的准确率到测试集骤降至88.94\%,后续可以考虑限制树的最大深度或增大叶节点的最小样本数来缓解。
	
	
	\section{总结与讨论}
	
	\subsection{研究结论}
	
	本文以LendingClub平台2007年以来的贷款记录为样本,经过系统性的探索分析和多模型对比实验,形成了以下几方面的主要结论。
	
	关于贷款违约的关键影响因素。探索性分析和Pearson相关性检验共同指向了若干对违约行为解释力较强的核心变量。其中贷款利率(\texttt{int\_rate})的区分力最为突出——高利率贷款的违约风险显著高于低利率贷款。从经济学角度看,高利率一方面反映了平台对该笔贷款信用风险的较高评估,另一方面也切实加重了借款人每月的偿债压力。贷款等级(\texttt{grade}/\texttt{sub\_grade})与违约率高度关联,从A等级到G等级呈单调递增趋势,说明平台评级体系在辨识信用风险方面确实有效。60个月期限贷款的违约率也远高于36个月,较长的偿还期限累积了更多不确定性。在借款人财务指标方面,DTI越高违约风险越大,年收入(\texttt{annual\_inc})越高风险越低,循环信用利用率(\texttt{revol\_util})偏高的借款人同样更容易违约。上述风险因子大致可以归为贷款属性(利率、等级、期限)和借款人财务状况(收入、DTI、循环信用利用率)两个维度,二者共同决定了违约的发生概率。
	
	关于模型预测性能的比较。三种分类器的实验数据表明,ANN在ROC-AUC上以0.905的得分遥遥领先,较XGBoost和随机森林分别高出约17和18个百分点,体现出深度网络在挖掘高维特征间复杂交互关系上的能力优势。同时ANN的泛化也最为稳健,训练与测试表现之间几乎没有落差。随机森林则面临比较严重的过拟合。值得注意的是,三种模型对违约类的召回率都在0.46至0.50之间徘徊,说明类别不均衡对模型训练的制约不容忽视。
	
	关于类别不均衡带来的影响。违约样本仅占总量的19.62\%,这种失衡使得三种模型在训练时都倾向于偏好多数类(全额偿还),以牺牲违约类识别灵敏度为代价换取整体准确率。然而在真实的风控场景中,漏放一个违约借款人造成的经济损失往往远超误拒一个优质客户所带来的业务损失,因此违约类召回率在实务应用中的优先级实际上更高。
	
	\subsection{分析方法总结}
	本文的数据分析工作覆盖了从数据探索到模型评估的完整链条,各环节采用的具体方法及对应目的汇总在表~\ref{tab:method_summary} 中。
	
	\begin{table}[htbp]
		\centering
		\caption{分析方法汇总}\label{tab:method_summary}
		\begin{tabular}{lll}
			\toprule
			分析阶段 & 具体方法 & 目的 \\
			\midrule
			描述性统计 & 均值、中位数、标准差、频率分布 & 把握数据基本特征 \\
			可视化分析 & 直方图、箱线图、计数图、热力图 & 直观展示分布与关联 \\
			相关性分析 & Pearson相关系数 & 量化变量间线性关系 \\
			缺失值处理 & 删除法、分组均值填补 & 保证数据完整性 \\
			异常值处理 & 基于业务逻辑的规则过滤 & 降低噪声干扰 \\
			特征编码 & 独热编码 & 类别变量数值化 \\
			数据标准化 & MinMax归一化 & 消除量纲差异 \\
			分类建模 & ANN、XGBoost、随机森林 & 贷款违约预测 \\
			模型评估 & Accuracy、Precision、Recall、F1、AUC、混淆矩阵 & 多维度性能评价 \\
			\bottomrule
		\end{tabular}
	\end{table}
	
	
	
	\subsection{讨论}
	
	模型选择方面,ANN在AUC上的优势与已有文献的发现基本一致\cite{kvamme2018predicting}。多层非线性映射使得神经网络有能力自动挖掘特征间的高阶交互效应,而树模型在这方面的表达能力相对受限。当然,ANN的"黑箱"特征也带来了可解释性方面的短板——在金融监管趋严的大环境下,模型的不可解释性可能成为其落地应用的一道障碍。另外需要指出的是,本文中XGBoost和随机森林均采用了默认或比较简单的超参数设置,并未做系统化调优。如果后续引入网格搜索或贝叶斯优化等手段对超参数空间做充分搜索,这两种模型的预测性能有望获得较大幅度的提升,和ANN之间的差距也可能因此缩小。
	
	类别不均衡问题是本研究中制约违约识别效果的一个核心瓶颈。围绕这一问题,后续工作可以从多个方向切入:过采样(如SMOTE\cite{chawla2002smote}、ADASYN)、欠采样(随机欠采样、Tomek Links等)、代价敏感学习(对不同类别设置差异化的误分类惩罚权重),以及专门针对不均衡数据设计的集成策略(如BalancedRandomForest、EasyEnsemble等)\cite{he2009learning}。合理引入上述策略,有望在保持总体准确率的前提下显著拉高违约类的召回率。
	
	特征工程方面,本文虽已完成缺失值填补、异常值剔除、独热编码和归一化等基本步骤,但仍有可以改进的空间。比如,引入L1正则化或借助树模型输出的特征重要性来做进一步的变量筛选,在降维的同时保留最关键的预测信息。又如,构造诸如利率与贷款金额的交叉项等交互特征来增强模型的表达能力。此外,目标编码(Target Encoding)等更精细的类别变量处理方式也值得尝试,以部分替代传统独热编码。
	
	在研究局限性方面,本文存在以下几点不足。第一,数据仅来自LendingClub一家平台,结论能否推广到其他P2P平台或更广泛的信贷市场,还需要更多实证支持。第二,受数据可得性的约束,本文没有纳入失业率、GDP增速等宏观经济指标,而这些系统性因素对贷款违约率可能有着不可忽视的影响。第三,XGBoost与随机森林的超参数调优不够充分,两类模型的潜力尚未被完全释放。第四,缺少对模型特征贡献的深入解读(例如基于SHAP方法\cite{lundberg2017unified}的分析),在模型透明度上仍有提升余地。
	
	\subsection{未来研究方向}
	
	结合以上结论和讨论,后续研究可沿若干方向展开。一是模型调优与新架构探索。借助贝叶斯优化等策略对超参数空间做更充分的搜索,同时尝试注意力机制、Transformer等较新的深度学习架构,看其能否捕获更为复杂的特征交互模式。二是不均衡学习策略的系统比较。对多种过采样、欠采样以及代价敏感学习方案进行逐一实验,以找到准确率和召回率之间的最佳平衡点。三是可解释性分析。运用SHAP等工具,量化各特征对模型输出的边际贡献,提升模型的透明度和可审计性,使之更好地满足金融监管对模型可解释性的要求。四是时序维度的建模。将贷款发放时间序列信息纳入分析框架,探索违约模式与宏观经济周期之间的关联。五是外部数据的融合。引入宏观经济指标乃至社交网络数据等外部信息源,拓宽特征空间,以期进一步提升预测效果。
	
\newpage
	\begin{thebibliography}{15}
		
		\bibitem{thomas2002credit}
		Thomas L C, Edelman D B, Crook J N. Credit Scoring and Its Applications[M]. Philadelphia: SIAM, 2002.
		
		\bibitem{lessmann2015benchmarking}
		Lessmann S, Baesens B, Seow H V, et al. Benchmarking state-of-the-art classification algorithms for credit scoring: An update of research[J]. European Journal of Operational Research, 2015, 247(1): 124-136.
		
		\bibitem{xia2017boosted}
		Xia Y, Liu C, Li Y, et al. A boosted decision tree approach using Bayesian hyper-parameter optimization for credit scoring[J]. Expert Systems with Applications, 2017, 78: 225-241.
		
		\bibitem{kvamme2018predicting}
		Kvamme H, Sellereite N, Aas K, et al. Predicting mortgage default using convolutional neural networks[J]. Expert Systems with Applications, 2018, 102: 207-217.
		
		\bibitem{serrano2016use}
		Serrano-Cinca C, Gutiérrez-Nieto B. The use of profit scoring as an alternative to credit scoring systems in peer-to-peer (P2P) lending[J]. Decision Support Systems, 2016, 89: 113-122.
		
		\bibitem{malekipirbazari2015risk}
		Malekipirbazari M, Aksakalli V. Risk assessment in social lending via random forests[J]. Expert Systems with Applications, 2015, 42(10): 4621-4631.
		
		\bibitem{chen2016xgboost}
		Chen T, Guestrin C. XGBoost: A scalable tree boosting system[C]//Proceedings of the 22nd ACM SIGKDD International Conference on Knowledge Discovery and Data Mining. 2016: 785-794.
		
		\bibitem{breiman2001random}
		Breiman L. Random forests[J]. Machine Learning, 2001, 45(1): 5-32.
		
		\bibitem{goodfellow2016deep}
		Goodfellow I, Bengio Y, Courville A. Deep Learning[M]. Cambridge: MIT Press, 2016.
		
		\bibitem{pedregosa2011scikit}
		Pedregosa F, Varoquaux G, Gramfort A, et al. Scikit-learn: Machine learning in Python[J]. Journal of Machine Learning Research, 2011, 12: 2825-2830.
		
		\bibitem{lendingclub_stats}
		LendingClub Corporation. LendingClub Statistics[EB/OL]. https://www.lendingclub.com/info/statistics.action.
		
		\bibitem{kaggle_lendingclub}
		Kaggle. All Lending Club Loan Data[EB/OL]. https://www.kaggle.com/datasets/wordsforthewise/lending-club/data.
		
		\bibitem{chawla2002smote}
		Chawla N V, Bowyer K W, Hall L O, et al. SMOTE: Synthetic minority over-sampling technique[J]. Journal of Artificial Intelligence Research, 2002, 16: 321-357.
		
		\bibitem{he2009learning}
		He H, Garcia E A. Learning from imbalanced data[J]. IEEE Transactions on Knowledge and Data Engineering, 2009, 21(9): 1263-1284.
		
		\bibitem{lundberg2017unified}
		Lundberg S M, Lee S I. A unified approach to interpreting model predictions[C]//Advances in Neural Information Processing Systems. 2017, 30: 4765-4774.
		
	\end{thebibliography}
	
\end{document}
