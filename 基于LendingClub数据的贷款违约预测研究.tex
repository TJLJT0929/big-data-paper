% !TEX program = xelatex
\documentclass[hyperref,a4paper,UTF8]{ctexart}

\usepackage[left=2.50cm, right=2.50cm, top=2.50cm, bottom=2.50cm]{geometry}

\usepackage{setspace}
\onehalfspacing

\usepackage[unicode=true,colorlinks,urlcolor=blue,linkcolor=blue,bookmarksnumbered=true]{hyperref}
\usepackage{latexsym,amssymb,amsmath,amsbsy,amsopn,amstext,amsthm,amsxtra,color,bm,calc,ifpdf}
\usepackage{graphicx}
\usepackage{subcaption}
\usepackage{enumerate}
\usepackage{fancyhdr}
\usepackage{algorithm}
\usepackage{algpseudocode}
\usepackage{makecell}
\usepackage{float}
\usepackage{listings}
\usepackage{multirow}
\usepackage[numbers,sort&compress]{natbib}
\usepackage{makeidx}
\usepackage{xcolor}
\usepackage{booktabs} 
\usepackage{fontspec}
\usepackage{hyperref}
\usepackage{pythonhighlight}
\graphicspath{{figures/}}


\newcommand{\upcite}[1]{\textsuperscript{\cite{#1}}}
\newtheorem{theorem}{Theorem}
\newtheorem{assumption}{Assumption}
\newtheorem{lemma}{Lemma}

\providecommand{\tightlist}{%
	\setlength{\itemsep}{0pt}\setlength{\parskip}{0pt}}

\pagestyle{fancy}
\fancyhead[L]{}
\fancyhead[C]{\fangsong 基于LendingClub平台数据的贷款违约预测研究}
\fancyhead[R]{}

\renewcommand{\abstractname}{\textbf{\large {摘\quad 要}}} 


\title{\textbf{{基于LendingClub平台数据的贷款违约预测研究}}\\
}
\author{
	\kaishu\normalsize
	姓名\ \underline{  刘建廷  } \qquad
	学号\ \underline{  251601032  } \qquad
	院系\ \underline{  中南大学商学院  }
}



\begin{document}
	
	\maketitle
	
	\vspace{0.5cm}
	
	\begin{abstract}
		P2P网络借贷平台的快速扩张使得信用风险管理问题日益突出。本文选取美国LendingClub平台公开的贷款数据集(约39.6万条记录、28个特征变量)作为研究样本,综合运用探索性数据分析、特征工程与多种机器学习算法,对借款人违约行为进行建模与预测。在探索性分析阶段,通过描述性统计与Pearson相关性检验,识别出贷款利率、信用等级、债务收入比及贷款期限等关键风险因子。在建模阶段,分别构建了人工神经网络(ANN)、极端梯度提升(XGBoost)和随机森林三种分类器,并从准确率、精确率、召回率、F1值及ROC曲线下面积(AUC)等多维度对模型性能进行评价。实验结果表明,ANN模型的测试集AUC达到0.905,显著高于XGBoost(0.734)与随机森林(0.725),在违约识别的综合判别能力上具有明显优势。本研究的结论可为P2P借贷平台优化风控策略、完善贷款审批流程提供定量参考。
	\end{abstract}
	
	\textbf{关键词:}贷款违约预测;LendingClub;人工神经网络;XGBoost;随机森林
	
	\section{引言}
	
	\subsection{研究背景}
	
	金融科技(FinTech)浪潮的持续推进对传统金融服务体系产生了深远影响。P2P(Peer-to-Peer)网络借贷作为其中的代表性业态,借助互联网平台直接连接资金供需双方,有效压缩了融资链条中的中介成本,提升了资金配置效率。LendingClub(借贷俱乐部)于2006年在美国旧金山成立,是全球规模最大的P2P借贷平台之一,也是率先在美国证券交易委员会(SEC)完成证券登记的P2P机构,其业务范围涵盖个人贷款、商业贷款及医疗融资等多个板块。
	
	然而,P2P模式在拓宽融资渠道的同时,也带来了不容忽视的信用风险。从平台运营角度而言,风险集中体现在两个层面:其一,误拒具有偿还能力的借款人将造成潜在业务流失;其二,向缺乏偿还能力的申请人发放贷款则直接导致平台及投资者蒙受损失。因此,如何有效甄别潜在违约群体、构建稳健的信用评估模型,始终是P2P借贷领域的核心议题。
	
	\subsection{研究现状}
	
	围绕贷款违约预测这一命题,国内外研究成果较为丰富。在传统统计方法层面,Logistic回归凭借模型透明度高、参数估计效率突出等特点,长期占据信用评分领域的主导地位\cite{thomas2002credit}。不过,当面临高维特征空间和复杂非线性映射关系时,传统线性���型的表达能力往往捉襟见肘。
	
	伴随机器学习理论与算法的不断成熟,众多学者开始探索将其引入信用风险评估场景。 Lessmann等\cite{lessmann2015benchmarking}针对多种分类算法在信用评分任务中的表现开展了系统性基准测试,实验证据表明集成学习方法(如随机森林、梯度提升树等)在预测精度上通常优于单一分类器。Xia等\cite{xia2017boosted}提出了一种融合贝叶斯超参数优化的XGBoost信用评分方案,在LendingClub数据集上展现了较强的判别能力。与此同时,深度学习技术也逐步渗透到该领域,Kvamme等\cite{kvamme2018predicting}的研究指出,神经网络结构在挖掘复杂非线性特征交互方面具有独特优势。
	
	在P2P借贷的实证研究中,LendingClub平台公开数据集凭借样本规模大、特征维度丰富、数据质量可控等特点,被学界广泛采用。Serrano-Cinca与Gutiérrez-Nieto\cite{serrano2016use}基于该数据集分析了违约行为的驱动因素,发现贷款等级、利率水平及借款人收入是关键预测变量。Malekipirbazari和Aksakalli\cite{malekipirbazari2015risk}对比了多种机器学习方法在该平台数据上的分类效果,随机森林在多项评价指标上占优。
	
	尽管既有研究积累了一定基础,但如下方面尚有拓展空间:第一,多数工作仅考察单一或少数模型,鲜有将传统树模型与深度神经网络进行系统对照;第二,不同研究在数据清洗与特征构造策略上差异较大,对异常值检测与处理的规范性不足;第三,关于类别不均衡问题的讨论尚不充分。
	
	\subsection{研究目的与意义}
	
	针对上述不足,本文以LendingClub平台贷款数据为实证样本,围绕以下四项目标展开研究。首先,通过系统的探索性数据分析,揭示贷款违约行为的主要驱动因素及其统计分布特征。其次,制定规范化的数据预处理与特征工程流程,构建适用于建模的高质量样本集。再次,分别构建人工神经网络、XGBoost和随机森林三种预测模型,并对其分类性能进行多指标比较。最后,为P2P借贷平台的贷款审批决策与风险防控提供数据驱动的参考依据。
	
	从理论层面看,本研究通过将深度学习模型纳入P2P信用评估的比较框架,丰富了该领域的方法论体系,为后续研究中的模型选择提供了实证参照。从实践层面看,研究成果可为LendingClub等平台的风险管控与决策优化提供技术支撑和策略参考。
	
	
	\section{数据来源与分析方法}
	
	\subsection{数据来源}
	
	本研究使用的数据集题为``All Lending Club loan data'',托管于Kaggle平台\cite{kaggle_lendingclub}。该数据集完整收录了LendingClub自2007年运营以来获批及被拒贷款的全部记录。
	
	数据集共包含约396{,}030条贷款记录和28个特征变量,覆盖面广泛,能够较为充分地反映该平台的信贷业务特征。其中,每条记录对应一笔贷款申请,因变量为贷款状态(\texttt{loan\_status}),取值分别为``Fully Paid''(全额偿还)和``Charged Off''(违约核销)。在原始样本中,全额偿还类占比约80.38\%,违约核销类占比约19.62\%,呈现出较为典型的类别不均衡分布。
	
	\subsection{数据特征描述}
	
	数据集所涵盖的28个自变量可按其业务含义归为三大类。
	
	\subsubsection{借款人基本信息特征}
	
	如表~\ref{tab:borrower_info} 所示,该类特征主要刻画借款人的就业状况、住房条件、收入水平及信用历史起始时间等基本画像信息。
	
	\begin{table}[htbp]
		\centering
		\caption{借款人基本信息特征}\label{tab:borrower_info}
		\begin{tabular}{ll}
			\toprule
			变量名 & 描述 \\
			\midrule
			\texttt{emp\_title}         & 借款人提供的职位名称 \\
			\texttt{emp\_length}        & 就业年限(0--10年,10表示十年及以上) \\
			\texttt{home\_ownership}    & 住房所有权状况(RENT/OWN/MORTGAGE/OTHER) \\
			\texttt{annual\_inc}        & 借款人自报年收入 \\
			\texttt{verification\_status} & 收入验证状态 \\
			\texttt{address}            & 借款人地址 \\
			\texttt{earliest\_cr\_line} & 最早信用额度开立月份 \\
			\bottomrule
		\end{tabular}
	\end{table}
	
	\subsubsection{贷款属性特征}
	
	如表~\ref{tab:loan_attr} 所示,该类特征涵盖贷款金额、期限、利率、信用等级等与贷款合约直接关联的属性。
	
	\begin{table}[htbp]
		\centering
		\caption{贷款属性特征}\label{tab:loan_attr}
		\begin{tabular}{ll}
			\toprule
			变量名 & 描述 \\
			\midrule
			\texttt{loan\_amnt}           & 贷款申请金额 \\
			\texttt{term}                 & 贷款期限(36个月或60个月) \\
			\texttt{int\_rate}            & 贷款利率 \\
			\texttt{installment}          & 月还款金额 \\
			\texttt{grade}                & 平台分配的贷款等级(A--G) \\
			\texttt{sub\_grade}           & 平台分配的贷款子等级 \\
			\texttt{issue\_d}             & 贷款发放月份 \\
			\texttt{purpose}              & 贷款用途类别 \\
			\texttt{title}                & 借款人提供的贷款标题 \\
			\texttt{initial\_list\_status} & 贷款初始上市状态(W/F) \\
			\texttt{application\_type}    & 申请类型(个人/联合申请) \\
			\bottomrule
		\end{tabular}
	\end{table}
	
	\subsubsection{借款人信用状况特征}
	
	如表~\ref{tab:credit_info} 所示,该类特征反映借款人在信用体系中的负债水平、账户结构及不良记录情况。
	
	\begin{table}[htbp]
		\centering
		\caption{借款人信用状况特征}\label{tab:credit_info}
		\begin{tabular}{ll}
			\toprule
			变量名 & 描述 \\
			\midrule
			\texttt{dti}                    & 债务收入比(月债务支出/月收入) \\
			\texttt{open\_acc}              & 开放信用额度数 \\
			\texttt{pub\_rec}               & 不良公共记录数 \\
			\texttt{revol\_bal}             & 循环信用总余额 \\
			\texttt{revol\_util}            & 循环信用利用率 \\
			\texttt{total\_acc}             & 总信用额度数 \\
			\texttt{mort\_acc}              & 抵押贷款账户数 \\
			\texttt{pub\_rec\_bankruptcies} & 公共记录中的破产次数 \\
			\bottomrule
		\end{tabular}
	\end{table}
	
	\subsection{研究方法与技术路线}
	
	本文的研究流程依次包含四个阶段。
	
	\textbf{第一阶段:探索性数据分析。}
	采用描述性统计、可视化手段(直方图、箱线图、热力图、计数图等)以及Pearson相关系数分析,系统考察各特征的分布形态、变量间的关联结构以及与因变量的关系强度。具体内容涵盖:因变量的类别分布分析、各自变量的单变量与双变量分析、特征间相关性热力图以及数值型特征与因变量的相关性排序。
	
	\textbf{第二阶段:数据预处理与特征工程。}
	针对原始数据中存在的质量问题,本文实施了一系列规范化的预处理操作。在缺失值方面,对缺失比例较低的特征(如\texttt{revol\_util}、\texttt{pub\_rec\_bankruptcies},缺失率均低于0.5\%),采取直接剔除含缺失值样本的策略;对缺失比例较高的特征(如\texttt{mort\_acc}),经相关性分析后选取与其关联度最强的变量(\texttt{total\_acc})按分组均值进行填补。在特征筛选方面,移除了唯一值过多而建模价值有限的变量(如\texttt{emp\_title}),以及可能引发数据泄露的时间变量(如\texttt{issue\_d});同时,鉴于各就业年限借款人的违约率几乎一致,\texttt{emp\_length}亦被删除,\texttt{title}与\texttt{purpose}存在信息冗余故仅保留后者。在异常值层面,依据探索性分析阶段的统计发现,在训练集中按既定规则剔除极端样本(如年收入超过250{,}000美元、DTI超过50等)。在特征编码方面,对\texttt{sub\_grade}、\texttt{verification\_status}、\texttt{purpose}等类别型变量执行独热编码。此外,还进行了若干特征转换操作:将\texttt{term}由字符串映射为数值型、从\texttt{address}中提取邮政编码、从\texttt{earliest\_cr\_line}中提取年份信息。最后,采用MinMaxScaler将全部特征归一化至$[0, 1]$区间,以消除不同量纲对模型训练的干扰。
	
	\textbf{第三阶段:模型构建与训练。}
	将数据集按67:33的比例划分为训练集(264{,}796条)与测试集(130{,}423条)。经异常值剔除后,训练集保留262{,}143条记录,最终特征维度为79维。本研究构建了三种分类模型。
	
	第一种为人工神经网络(ANN),基于Keras框架搭建前馈神经网络,网络结构包含3个隐藏层,每层设置150个神经元,隐藏层激活函数采用ReLU,输出层采用Sigmoid函数以输出违约概率。训练过程中引入批标准化(Batch Normalization)及Dropout(丢弃率0.1)以抑制过拟合现象。优化算法选用Adam(学习率0.001),损失函数为二元交叉熵,共训练25个轮次,小批量大小设为256。第二种为XGBoost分类器,调用XGBoost库中的\texttt{XGBClassifier}接口,采用默认超参数配置进行训练。XGBoost是一种基于梯度提升框架的集成学习方法,其核心思想在于通过逐轮构建决策树拟合前一轮的残差,最终将各弱学习器的输出加以集成\cite{chen2016xgboost}。第三种为随机森林分类器,使用Scikit-learn中的\texttt{RandomForestClassifier},基学习器数量设为100棵。随机森林通过自助采样(Bootstrap Sampling)与随机特征子集选取构建决策树集合,并以投票方式确定最终预测类别\cite{breiman2001random}。
	
	\textbf{第四阶段:模型评估与比较。}
	为全面评判三种模型的分类效果,本文采用了多维度的指标体系。准确率(Accuracy)反映被正确分类的样本占总样本的比例;精确率(Precision)衡量模型判定为正类的样本中真正为正类的比例;召回率(Recall)度量实际正类样本中被模型正确识别的比例;F1值作为精确率与召回率的调和平均数,兼顾了两者的平衡;ROC-AUC即受试者工作特征曲线下面积,能够综合度量模型对正负样本的区分能力,在类别不均衡场景下尤为重要。此外,本文还借助混淆矩阵以矩阵形式直观呈现各类别的预测结果,便于细粒度分析模型在不同类别上的表现。
	
	\subsection{实验环境}
	
	本研究的主要实验工具与软件包如下:编程语言为Python 3;数据处理依赖Pandas与NumPy;可视化工具包括Matplotlib、Seaborn和hvPlot;机器学习部分使用Scikit-learn\cite{pedregosa2011scikit}与XGBoost;深度学习框架为TensorFlow/Keras;运行平台为Kaggle Notebook。
	
	
	\section{研究结果与分析}
	
	\subsection{探索性数据分析结果}
	
	\subsubsection{因变量分布}
	
	对因变量\texttt{loan\_status}的频率统计显示,全额偿还(Fully Paid)样本占总体的80.38\%,违约核销(Charged Off)样本占19.62\%。非违约样本数量远超违约样本,二者之比约为4:1,呈现出信用风险建模中常见的类别不均衡特征。这一分布特点意味着,如果模型简单地将所有样本预测为多数类(全额偿还),也能获得约80\%的准确率,因此在后续建模过程中需引入更具区分力的评价指标(如ROC-AUC),并充分考虑类别不均衡对模型训练的影响。

	
	\subsubsection{特征间相关性分析}
	
	借助Pearson相关系数热力图(如图~\ref{fig:corr_heatmap} 所示)对数值型特征间的线性关联强度进行检验,主要发现如下:
	
	贷款金额(\texttt{loan\_amnt})与月还款额(\texttt{installment})呈近乎完全正相关(相关系数趋近于1.0),这一结果符合业务常识——月还款额由贷款金额、利率和期限共同决定,因而二者之间存在高度冗余。贷款金额与贷款利率之间同样呈正相关,即金额较大的贷款往往伴随更高的利率定价。开放信用额度数(\texttt{open\_acc})与总信用额度数(\texttt{total\_acc})之间具有较强的正向关联。值得注意的是,抵押贷款账户数(\texttt{mort\_acc})与总信用额度数的相关性最为突出,这一发现在后续缺失值填补策略的设计中得到了应用。
	
	\begin{figure}[H]
		\centering
		\includegraphics[width=0.9\textwidth]{Figure/2}
		\caption{数值型特征间 Pearson 相关系数热力图}
		\label{fig:corr_heatmap}
	\end{figure}
	
	\subsubsection{贷款金额与月还款额分析}
	
	按贷款状态对贷款金额和月还款额进行分层考察,结果如图~\ref{fig:loan_amnt_installment} 所示。从直方图和箱线图中均可观察到,违约样本的贷款金额中位数及均值略高于全额偿还样本,提示较高的贷款金额可能与违约风险的上升存在关联。月还款额的分布趋势与贷款金额基本一致,二者的分层分布形态进一步印证了它们之间的高度相关性。具体而言,违约组的贷款金额箱线图在四分位间距和极值范围上均高于全额偿还组,说明违约借款人倾向于申请更高金额的贷款。
	


\begin{figure}[H]
	\centering
	
	\begin{subfigure}[b]{0.48\textwidth}
		\centering
		\includegraphics[width=\linewidth]{Figure/3a}
		\caption{贷款金额直方图}
		\label{fig:loan_amnt_hist}
	\end{subfigure}
	\hfill
	\begin{subfigure}[b]{0.48\textwidth}
		\centering
		\includegraphics[width=\linewidth]{Figure/3b}
		\caption{月还款额直方图}
		\label{fig:installment_hist}
	\end{subfigure}
	
	\vspace{0.5cm}
	
	\begin{subfigure}[b]{0.42\textwidth}
		\centering
		\includegraphics[width=\linewidth]{Figure/3c}
		\caption{贷款金额箱线图}
		\label{fig:loan_amnt_box}
	\end{subfigure}
	\hfill
	\begin{subfigure}[b]{0.42\textwidth}
		\centering
		\includegraphics[width=\linewidth]{Figure/3d}
		\caption{月还款额箱线图}
		\label{fig:installment_box}
	\end{subfigure}
	
	\caption{贷款金额与月还款额按贷款状态的分布}
	\label{fig:loan_amnt_installment}
\end{figure}
	
	\subsubsection{贷款等级与子等级分析}
	
	LendingClub依据借款人信用状况将贷款划分为A(最优)至G(最差)七个等级。图~\ref{fig:grade_subgrade_a} 展示了按等级和子等级划分的贷款状态分布,统计结果揭示了清晰的违约梯度规律:从A至G等级,违约率呈单调递增态势。A等级贷款的违约占比最低,而F、G等级贷款的违约占比最高。图~\ref{fig:grade_subgrade_b} 进一步聚焦于F、G子等级内部的分布情况,可以发现违约样本与全额偿还样本的数量差距明显收窄,部分子等级的违约率已接近50\%。上述结果表明,LendingClub的内部评级体系对借款人违约风险具有较好的区分效力,评级信息对信用风险建模具有重要的参考价值。
	
\begin{figure}[H]
	\centering
	
	\begin{subfigure}[b]{0.4\textwidth}
		\centering
		\includegraphics[width=\linewidth]{Figure/4a}
		\caption{按贷款等级与子等级划分的贷款状态分布}
		\label{fig:grade_subgrade_a}
	\end{subfigure}
	\hfill
	\begin{subfigure}[b]{0.45\textwidth}
		\centering
		\includegraphics[width=\linewidth]{Figure/4b}
		\caption{F和G等级贷款的违约分布}
		\label{fig:grade_subgrade_b}
	\end{subfigure}
	
	\caption{贷款等级与违约状态分析}
	\label{fig:grade_subgrade}
\end{figure}
	
	\subsubsection{贷款期限、住房状况、收入验证与贷款用途分析}
	
	图~\ref{fig:term_home_verify_purpose} 分别展示了贷款期限、住房所有权、收入验证状态和贷款用途四个类别特征按贷款状态的分布情况。就贷款期限而言,60个月期限贷款的违约率显著高于36个月期限贷款,说明较长的还款周期伴随着更多的不确定性,增加了违约发生的可能性。就住房所有权而言,租房(RENT)和按揭(MORTGAGE)类借款人构成了样本的主体部分,不同住房类别间的违约率差异并不显著;值得注意的是,``ANY''和``NONE''两类样本极少且缺乏统计稳定性,本研究将其合并至``OTHER''类别以提高分析的可靠性。就收入验证状态而言,已验证(Verified)、来源已验证(Source Verified)与未验证(Not Verified)三类借款人的违约比例相近,表明收入验证状态对违约预测的边际贡献有限。就贷款用途而言,债务整合(debt\_consolidation)是最为普遍的借款目的,其次为信用卡偿还(credit\_card),不同用途类别间的违约率存在一定差异,但总体分化程度不及利率和信用等级等特征。
	
	\begin{figure}[H]
		\centering
		\includegraphics[width=0.85\textwidth]{Figure/5}
		\caption{贷款期限、住房状况、收入验证状态与贷款用途按贷款状态的分布}
		\label{fig:term_home_verify_purpose}
	\end{figure}
	
	\subsubsection{利率与年收入分析}
	
	贷款利率和年收入是反映借款人信用风险水平的两个核心变量。如图~\ref{fig:int_rate} 所示,违约样本的利率分布明显偏向高利率区间,而全额偿还样本则集中于较低利率区间,二者的分布差异十分显著,表明利率水平是违约预测中最为重要的判别变量之一。这一现象的经济学解释在于:高利率通常对应较高的信用风险等级,同时也加重了借款人的月度偿还负担,双重因素共同推高了违约概率。
	
	如图~\ref{fig:annual_inc} 所示,年收入低于250{,}000美元的借款人占据了样本的绝大多数。进一步统计表明,年收入超过1{,}000{,}000美元的借款人仅有75人(不足样本总量的0.02\%),而超过250{,}000美元的共计4{,}077人。总体而言,高收入群体的违约率相对较低,但由于该群体的样本量过小,其统计推断的可靠性受到一定限制。
	
\begin{figure}[htbp]
	\centering
	
	\begin{subfigure}[b]{0.4\textwidth}
		\centering
		\includegraphics[width=\linewidth]{Figure/6a}
		\caption{贷款利率按贷款状态的分布直方图}
		\label{fig:int_rate}
	\end{subfigure}
	\hfill
	\begin{subfigure}[b]{0.55\textwidth}
		\centering
		\includegraphics[width=\linewidth]{Figure/6b}
		\caption{年收入($\leq$250{,}000美元)按贷款状态的分布直方图}
		\label{fig:annual_inc}
	\end{subfigure}
	
	\caption{贷款利率与年收入按贷款状态的分布}
	\label{fig:int_rate_annual_inc}
\end{figure}
	
	\subsubsection{债务收入比与循环信用分析}
	
	图~\ref{fig:dti_revol} 展示了债务收入比、循环信用利用率及循环信用余额按贷款状态的分布情况。就债务收入比而言,随着DTI值的升高,借款人的违约倾向有所增强。DTI超过50的极端记录数量极少(仅占总样本的很小比例),这类异常值在后续建模阶段予以剔除。就循环信用利用率而言,较高的利用率与违约风险的提升具有正向关联,利用率超过120\%的记录极为罕见,同样作为异常值处理。就循环信用余额而言,余额超过250{,}000美元的样本极少,其中亦包含违约记录。上述分析结果表明,借款人的负债水平和信用使用状况对违约行为具有显著的解释力,可作为风险评估的重要参考维度。
	
\begin{figure}[H]
	\centering
	\begin{subfigure}[b]{0.32\textwidth}
		\centering
		\includegraphics[width=\linewidth]{Figure/7a}
	\end{subfigure}
	\hfill
	\begin{subfigure}[b]{0.32\textwidth}
		\centering
		\includegraphics[width=\linewidth]{Figure/7b}
	\end{subfigure}
	\hfill
	\begin{subfigure}[b]{0.32\textwidth}
		\centering
		\includegraphics[width=\linewidth]{Figure/7c}
	\end{subfigure}
	
	\caption{债务收入比、循环信用利用率及循环信用余额按贷款状态的分布}
	\label{fig:dti_revol}
\end{figure}

	
	\subsubsection{公共记录与破产记录分析}
	
	图~\ref{fig:pub_rec_bankruptcy} 展示了不良公共记录、初始上市状态、申请类型、破产记录和抵押贷款账户数等变量按贷款状态的分布情况。就不良公共记录而言,绝大多数借款人不存在不良记录(\texttt{pub\_rec}=0),而有不良记录的借款人违约率偏高。破产记录的分布与之类似,大部分借款人无破产历史,有破产记录者的违约风险略有抬升。基于上述发现,在特征工程阶段将不良记录和破产记录分别转化为二值变量(0或1),以简化特征表达并突出其有无对违约概率的影响。就初始上市状态和申请类型而言,不同类别间的违约率分化程度相对有限。
	
	\begin{figure}[H]
		\centering
		\includegraphics[width=0.9\textwidth]{Figure/8}
		\caption{公共记录、初始上市状态、申请类型、破产记录与抵押贷款账户按贷款状态的分布}
		\label{fig:pub_rec_bankruptcy}
	\end{figure}
	
	\subsubsection{数值特征与因变量的相关性排序}
	
	计算全部数值型特征与因变量之间的Pearson相关系数并降序排列,结果如图~\ref{fig:feature_corr_target} 所示。在与偿还行为正相关的特征中,抵押贷款账户数和年收入的正向效应较为微弱,表明仅凭单一收入或资产指标尚不足以有效区分违约与非违约群体。在与违约行为正相关(即与偿还行为负相关)的特征中,利率(\texttt{int\_rate})的负相关程度最为突出,其后依次为贷款期限(\texttt{term})和债务收入比(\texttt{dti})。这一排序结果进一步验证了前文探索性分析中的定性发现,也为后续模型构建中特征的筛选与解读提供了定量依据。
	
	\begin{figure}[H]
		\centering
		\includegraphics[width=0.9\textwidth]{Figure/9}
		\caption{数值型特征与贷款状态(因变量)的Pearson相关系数排序}
		\label{fig:feature_corr_target}
	\end{figure}
	
	\subsection{数据预处理结果}
	
	\subsubsection{缺失值处理}
	
	原始数据中存在缺失值的特征及对应处理方式汇总于表~\ref{tab:missing}。
	
	\begin{table}[H]
		\centering
		\caption{缺失值处理方案}\label{tab:missing}
		\begin{tabular}{llll}
			\toprule
			特征 & 缺失数量 & 缺失比例 & 处理方式 \\
			\midrule
			\texttt{emp\_title}            & 较多 & --- & 删除该特征(唯一值过多,不适合编码) \\
			\texttt{emp\_length}           & 较多 & --- & 删除该特征(各就业年限违约率几乎一致) \\
			\texttt{title}                 & 少量 & --- & 删除该特征(与\texttt{purpose}信息重叠) \\
			\texttt{mort\_acc}             & 较多 & --- & 按\texttt{total\_acc}分组均值填补 \\
			\texttt{revol\_util}           & 少量 & $<$0.5\% & 删除含缺失值的行 \\
			\texttt{pub\_rec\_bankruptcies} & 少量 & $<$0.5\% & 删除含缺失值的行 \\
			\bottomrule
		\end{tabular}
	\end{table}
	
	经上述处理后,数据集保留了绝大部分样本,信息损失可忽略不计。
	
	\subsubsection{特征工程}
	
	经过特征工程处理后,数据集最终包含79个特征维度(含独热编码产生的虚拟变量)。具体而言,首先将\texttt{term}由字符型(`` 36 months''/`` 60 months'')映射为数值(36/60),以便模型直接利用其数量信息。其次,由于\texttt{sub\_grade}已完整包含\texttt{grade}的信息(子等级是等级的细分),故删除\texttt{grade}以避免特征冗余。对\texttt{sub\_grade}、\texttt{verification\_status}、\texttt{purpose}、\texttt{initial\_list\_status}、\texttt{application\_type}、\texttt{home\_ownership}等类别型变量进行独热编码,并设置drop\_first=True以规避多重共线性问题。从\texttt{address}字段中提取邮政编码并进行独热编码,从\texttt{earliest\_cr\_line}中提取年份信息作为数值特征。同时,删除\texttt{issue\_d}以防止因使用未来信息而产生数据泄露。此外,将\texttt{pub\_rec}、\texttt{mort\_acc}、\texttt{pub\_rec\_bankruptcies}转化为二值变量(0表示无记录、1表示有记录),以简化特征表达并突出其"有无"属性对违约概率的影响。
	
	\subsubsection{异常值处理}
	
	基于探索性分析阶段的统计发现,在训练集中按表~\ref{tab:outlier} 所列规则剔除异常样本。
	
	\begin{table}[htbp]
		\centering
		\caption{异常值剔除规则}\label{tab:outlier}
		\begin{tabular}{ll}
			\toprule
			过滤条件 & 说明 \\
			\midrule
			\texttt{annual\_inc} $\leq$ 250{,}000    & 移除极端高收入样本 \\
			\texttt{dti} $\leq$ 50                    & 移除极端高债务收入比样本 \\
			\texttt{open\_acc} $\leq$ 40              & 移除开放信用账户数过多的样本 \\
			\texttt{total\_acc} $\leq$ 80             & 移除总信用账户数过多的样本 \\
			\texttt{revol\_util} $\leq$ 120           & 移除循环信用利用率过高的样本 \\
			\texttt{revol\_bal} $\leq$ 250{,}000      & 移除循环信用余额过高的样本 \\
			\bottomrule
		\end{tabular}
	\end{table}
	
	需要指出的是,异常值剔除仅在训练集上实施,测试集保持原始分布不变,以便更客观地评价模型在真实数据环境下的泛化表现。处理后训练集由264{,}796条缩减至262{,}143条,剔除比例约为1.0\%。
	
	\subsection{模型预测结果}
	
\subsubsection{人工神经网络模型结果}

人工神经网络(ANN)模型在训练集与测试集上均表现出稳定的分类性能,训练集准确率为88.84\%,测试集准确率为88.87\%,未见明显过拟合。该模型对全额偿还类(类别1)具有突出的识别能力,召回率达0.99,但对违约类(类别0)的召回率较低,仅为0.46,反映出类别不均衡的影响。两类别的F1值分别为0.62(违约类)与0.93(全额偿还类)。从混淆矩阵可见,模型对违约类的预测存在较多漏判:训练集中实际违约样本共51,665例,其中27,985例被误判为偿还;测试集中实际违约样本25,480例,误判数为13,798例。

图~\ref{fig:ann_training}展示了ANN模型在训练过程中损失函数值和AUC指标随轮次的演化曲线。训练损失在前几个轮次内迅速下降并趋于平稳,AUC指标则同步上升并收敛,说明模型的学习过程稳定且有效。ANN模型的ROC-AUC得分为0.905,在三种模型中居于首位,表明其在正负样本综合判别方面具有较强的能力。

\begin{figure}[H]
	\centering
	\includegraphics[width=0.9\textwidth]{Figure/10}
	\caption{ANN模型训练过程中的损失函数值与AUC指标演化曲线}
	\label{fig:ann_training}
\end{figure}
	
\subsubsection{XGBoost分类器结果}

XGBoost模型在训练集与测试集上的表现总体稳健。训练集准确率为89.60\%,测试集准确率为88.94\%,二者差距微小,表明模型未出现显著过拟合,泛化能力良好。其在违约类(类别0)的识别上略有改善,训练集与测试集的召回率分别为0.50与0.48,均略高于前述ANN模型,意味着对少数类的捕捉能力稍强。该模型对全额偿还类(类别1)的识别依然保持高召回率(0.99)。然而,其综合判别性能指标ROC-AUC得分为0.734,显著低于ANN模型,说明其在权衡正负样本分类性能方面存在不足。

训练集混淆矩阵显示,在51,665个实际违约样本中,模型正确识别了25,828个,同时误判了25,837个为偿还类。测试集的详细分类情况可通过图~\ref{fig:xgb_cm}所示的混淆矩阵进行直观审视。
	
\subsubsection{随机森林分类器结果}

随机森林模型表现出明显的过拟合现象。其在训练集上达到了完美的分类结果(准确率100\%,各类别精确率、召回率及F1值均为1.00),训练集混淆矩阵显示所有样本均被正确识别。然而,在测试集上,其性能显著下降,准确率为88.94\%。测试集上对违约类(类别0)的召回率仅为0.46,与ANN模型持平,但对违约类的预测精确率较高(0.96)。该模型对多数类(全额偿还类)依然保持高召回率(0.99)。其综合性能指标ROC-AUC得分为0.725,为三种模型中最低。训练集与测试集性能的巨大落差,表明模型在训练时可能因树结构过于复杂而过度记忆了训练数据,导致泛化能力受限。测试集的具体误判情况可通过图~\ref{fig:rf_cm}所示的混淆矩阵进行观察。

\begin{figure}[H]
	\centering
	\begin{minipage}{0.48\textwidth}
		\centering
		\includegraphics[width=\linewidth]{Figure/11}
		\caption{XGBoost模型测试集混淆矩阵}
		\label{fig:xgb_cm}
	\end{minipage}
	\hfill
	\begin{minipage}{0.48\textwidth}
		\centering
		\includegraphics[width=\linewidth]{Figure/12}
		\caption{随机森林模型测试集混淆矩阵}
		\label{fig:rf_cm}
	\end{minipage}
\end{figure}

	
	\subsubsection{模型综合比较}
	
	三种模型在测试集上的核心指标汇总于表~\ref{tab:model_compare},图~\ref{fig:model_compare}以 柱状图的形式直观对比了三种模型在训练集与测试集上的ROC-AUC得分差异。
	
	\begin{table}[htbp]
		\centering
		\caption{三种模型性能综合对比}\label{tab:model_compare}
		\begin{tabular}{lccc}
			\toprule
			模型 & 训练集准确率 & 测试集准确率 & 测试集ROC-AUC \\
			\midrule
			人工神经网络(ANN) & 88.84\% & 88.87\% & \textbf{0.905} \\
			XGBoost             & 89.60\% & 88.94\% & 0.734 \\
			随机森林            & 100.00\% & 88.94\% & 0.725 \\
			\bottomrule
		\end{tabular}
	\end{table}
	
	\begin{figure}[H]
		\centering
		\includegraphics[width=0.75\textwidth]{Figure/13}
		\caption{三种模型在训练集与测试集上的ROC-AUC得分对比}
		\label{fig:model_compare}
	\end{figure}
	
	从ROC-AUC的角度分析,ANN以0.905的得分大幅领先于XGBoost(0.734)和随机森林(0.725),二者之间的差距分别达到17.1和18.0个百分点,说明神经网络在正负样本的综合判别能力上占有显著优势,能够更充分地捕捉贷款数据中的非线性特征交互关系。
	
	从准确率的角度分析,三种模型的测试集准确率均在89\%附近,差异甚微。然而,鉴于数据集中非违约样本占比高达80\%,单纯依赖准确率评判模型优劣并不充分。一个始终将所有样本预测为"全额偿还"的朴素分类器也能达到约80\%的准确率,因此ROC-AUC才是更具参考价值的综合性指标。
	
	从过拟合程度分析,ANN模型的训练与测试性能几乎一致,泛化能力最为稳健,这得益于其训练过程中引入的批标准化和Dropout正则化机制。XGBoost略有过拟合但幅度可控,说明梯度提升框架的内在正则化特性(如树的深度限制和学习率衰减)发挥了一定作用。随机森林训练集准确率达100\%而测试集骤降至88.94\%,过拟合问题最为突出,后续可通过限制树深度、增大叶节点最小样本数等超参数调整手段加以缓解。
	
	
	\section{总结与讨论}
	
	\subsection{研究结论}
	
	本文以LendingClub平台2007年以来的贷款记录为样本,在系统的探索性分析和多模型对比实验的基础上,得出以下主要结论。
	
	在贷款违约的关键驱动因素方面,经探索性分析与Pearson相关性检验,本文识别出若干对违约行为具有显著解释力的核心变量。贷款利率(\texttt{int\_rate})是最具区分力的违约预测因子,高利率贷款的违约风险显著偏高,其经济学逻辑在于高利率通常对应较高的信用风险等级,同时也加重了借款人的月度偿还负担。贷款等级(\texttt{grade}/\texttt{sub\_grade})与违约率之间表现出高度关联,从A等级到G等级违约率单调递增,充分反映出平台评级体系在信用风险辨识方面的有效性。贷款期限(\texttt{term})对违约概率亦有显著影响,60个月期限贷款的违约率远高于36个月期限贷款,较长的还款周期意味着更大的不确定性累积。此外,债务收入比(\texttt{dti})与违约风险呈正相关,年收入(\texttt{annual\_inc})与违约风险呈负相关,循环信用利用率(\texttt{revol\_util})较高的借款人违约风险更大。上述因素可概括为贷款属性维度(利率、等级、期限等)与借款人财务维度(收入、DTI、循环信用利用率等),二者共同构成了违约概率的决定性因素群。
	
	在模型性能对比方面,三种分类器的实验结果表明,ANN在ROC-AUC指标上显著优于XGBoost和随机森林,体现了深度神经网络在挖掘高维数据中复杂特征交互关系方面的能力优势。ANN的测试集AUC为0.905,分别高出XGBoost和随机森林17.1与18.0个百分点,且其泛化能力最为稳健,训练集与测试集性能几乎无差异。相比之下,随机森林模型面临较为严重的过拟合问题。不过,三种模型在违约类的召回率上均不够理想(约0.46--0.50),凸显了类别不均衡对模型训练的制约作用。
	
	在类别不均衡的影响方面,违约样本仅占总体的19.62\%,这一失衡导致所有模型在训练过程中倾向于将样本判定为多数类(全额偿还),从而牺牲了对少数类(违约)的识别灵敏度。在实际风控场景中,漏放一位违约借款人所造成的经济损失往往远大于误拒一位优质借款人带来的业务损失,因而违约类的召回率在应用中具有更高的优先级。
	
	\subsection{分析方法总结}
	本研究的数据分析流程贯穿数据探索、预处理、建模与评估四个环节,各阶段所采用的具体方法及其目的汇总于表~\ref{tab:method_summary}。
	
	\begin{table}[htbp]
		\centering
		\caption{分析方法汇总}\label{tab:method_summary}
		\begin{tabular}{lll}
			\toprule
			分析阶段 & 具体方法 & 目的 \\
			\midrule
			描述性统计 & 均值、中位数、标准差、频率分布 & 把握数据基本特征 \\
			可视化分析 & 直方图、箱线图、计数图、热力图 & 直观展示分布与关联 \\
			相关性分析 & Pearson相关系数 & 量化变量间线性关系 \\
			缺失值处理 & 删除法、分组均值填补 & 保证数据完整性 \\
			异常值处理 & 基于业务逻辑的规则过滤 & 降低噪声干扰 \\
			特征编码 & 独热编码 & 类别变量数值化 \\
			数据标准化 & MinMax归一化 & 消除量纲差异 \\
			分类建模 & ANN、XGBoost、随机森林 & 贷款违约预测 \\
			模型评估 & Accuracy、Precision、Recall、F1、AUC、混淆矩阵 & 多维度性能评价 \\
			\bottomrule
		\end{tabular}
	\end{table}
	
	
	
	\subsection{讨论}
	
	关于模型选择问题,ANN在AUC指标上的优势与既有文献的结论相吻合\cite{kvamme2018predicting}。神经网络通过多层非线性映射能够自动发现特征间的高阶交互,而传统树模型在这一方面的能力相对有限。然而,ANN的``黑箱''属性也带来了可解释性不足的问题,在金融监管日趋严格的背景下,这一缺陷可能制约其在实际审批流程中的应用落地。此外,本研究中XGBoost和随机森林均沿用了默认或较简单的超参数配置,尚未经过系统调优。若引入网格搜索或贝叶斯优化等方法对超参数空间进行充分探索,这两种模型的性能有望获得显著提升,与ANN之间的差距也可能相应缩小。
	
	关于类别不均衡处理问题,类别不均衡是本研究中制约违约识别效果的关键因素。后续研究可从多个方向尝试改进,包括过采样方法(如SMOTE\cite{chawla2002smote}、ADASYN等)、欠采样方法(如随机欠采样、Tomek Links等)、代价敏感学习(为不同类别赋予差异化的误分类代价权重)以及专门面向不均衡数据的集成方法(如BalancedRandomForest、EasyEnsemble等)\cite{he2009learning}。这些策略的引入有望在保持整体准确率的同时,显著提升违约类的召回率。
	
	关于特征工程的改进空间,本文在特征工程方面已覆盖缺失值填补、异常值剔除、独热编码及归一化等环节,但仍存在优化余地。例如,可引入L1正则化或基于树模型的特征重要性进行特征筛选,从而在降低维度的同时保留最具判别力的变量。此外,可构建交互特征(如利率与贷款金额的乘积项)以增强模型表达力,也可尝试目标编码(Target Encoding)等更精细的类别特征处理策略来替代传统的独热编码。
	
	关于研究局限性,本研究主要存在如下不足。首先,样本仅覆盖LendingClub一家平台,结论的外推性有待在其他平台或市场上加以验证。其次,受数据可得性限制,未引入宏观经济指标(如失业率、GDP增速等)作为外部特征,而这类系统性因素对贷款违约率可能具有重要影响。再次,XGBoost和随机森林的超参数优化尚不充分,未充分释放这两种模型的潜力。最后,未对模型的特征贡献进行深入的可解释性分析(如SHAP方法\cite{lundberg2017unified}),在模型透明度方面存在提升空间。
	
	\subsection{未来研究方向}
	
	基于上述结论与讨论,后续工作可从以下方向拓展。在模型调优与架构探索方面,可采用贝叶斯优化等方法系统搜索超参数空间,充分释放各模型的潜力,同时尝试引入注意力机制、Transformer等更先进的深度学习架构以捕获更复杂的特征交互模式。在不均衡学习策略方面,可系统评估各类过采样、欠采样及代价敏感学习策略对违约识别效果的提升幅度,寻找准确率与召回率之间的最优平衡点。在可解释性分析方面,可利用SHAP等工具量化各特征对预测结果的边际贡献,增强模型的透明度与可审计性,使其更好地服务于金融监管要求。在时序维度建模方面,可结合贷款发放时间序列信息,挖掘违约模式随宏观经济周期的演化规律。在外部数据融合方面,可纳入宏观经济指标、社会网络数据等外部信息源,拓展特征空间,以期进一步提升预测精度。
	
\newpage
	\begin{thebibliography}{15}
		
		\bibitem{thomas2002credit}
		Thomas L C, Edelman D B, Crook J N. Credit Scoring and Its Applications[M]. Philadelphia: SIAM, 2002.
		
		\bibitem{lessmann2015benchmarking}
		Lessmann S, Baesens B, Seow H V, et al. Benchmarking state-of-the-art classification algorithms for credit scoring: An update of research[J]. European Journal of Operational Research, 2015, 247(1): 124-136.
		
		\bibitem{xia2017boosted}
		Xia Y, Liu C, Li Y, et al. A boosted decision tree approach using Bayesian hyper-parameter optimization for credit scoring[J]. Expert Systems with Applications, 2017, 78: 225-241.
		
		\bibitem{kvamme2018predicting}
		Kvamme H, Sellereite N, Aas K, et al. Predicting mortgage default using convolutional neural networks[J]. Expert Systems with Applications, 2018, 102: 207-217.
		
		\bibitem{serrano2016use}
		Serrano-Cinca C, Gutiérrez-Nieto B. The use of profit scoring as an alternative to credit scoring systems in peer-to-peer (P2P) lending[J]. Decision Support Systems, 2016, 89: 113-122.
		
		\bibitem{malekipirbazari2015risk}
		Malekipirbazari M, Aksakalli V. Risk assessment in social lending via random forests[J]. Expert Systems with Applications, 2015, 42(10): 4621-4631.
		
		\bibitem{chen2016xgboost}
		Chen T, Guestrin C. XGBoost: A scalable tree boosting system[C]//Proceedings of the 22nd ACM SIGKDD International Conference on Knowledge Discovery and Data Mining. 2016: 785-794.
		
		\bibitem{breiman2001random}
		Breiman L. Random forests[J]. Machine Learning, 2001, 45(1): 5-32.
		
		\bibitem{goodfellow2016deep}
		Goodfellow I, Bengio Y, Courville A. Deep Learning[M]. Cambridge: MIT Press, 2016.
		
		\bibitem{pedregosa2011scikit}
		Pedregosa F, Varoquaux G, Gramfort A, et al. Scikit-learn: Machine learning in Python[J]. Journal of Machine Learning Research, 2011, 12: 2825-2830.
		
		\bibitem{lendingclub_stats}
		LendingClub Corporation. LendingClub Statistics[EB/OL]. https://www.lendingclub.com/info/statistics.action.
		
		\bibitem{kaggle_lendingclub}
		Kaggle. All Lending Club Loan Data[EB/OL]. https://www.kaggle.com/datasets/wordsforthewise/lending-club/data.
		
		\bibitem{chawla2002smote}
		Chawla N V, Bowyer K W, Hall L O, et al. SMOTE: Synthetic minority over-sampling technique[J]. Journal of Artificial Intelligence Research, 2002, 16: 321-357.
		
		\bibitem{he2009learning}
		He H, Garcia E A. Learning from imbalanced data[J]. IEEE Transactions on Knowledge and Data Engineering, 2009, 21(9): 1263-1284.
		
		\bibitem{lundberg2017unified}
		Lundberg S M, Lee S I. A unified approach to interpreting model predictions[C]//Advances in Neural Information Processing Systems. 2017, 30: 4765-4774.
		
	\end{thebibliography}
	
\end{document}
